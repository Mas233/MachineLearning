\documentclass{article}
\usepackage{ctex}
\usepackage{graphicx}
\usepackage{hyperref}

\title{神经网络实验:LeNet-5}
\author{王铭嵩}
\date{\today}

\begin{document}
\maketitle
\newpage
\section{神经网络基本原理}
卷积神经网络主要由卷积层、池化层、全连接层构成。卷积层通过对输入图像进行卷积操作来提取图像特征。池化层对输入的特征图片进行压缩,简化网络计算复杂度。全连接层连接所有的特征,并将输出值送给分类器。

\section{LeNet-5基本结构}
\subsection{输入层}
用以接收输入的图像数据。在CIFAR10数据集上,输入为32x32x3的图像数据。图像长宽为32,以及RGB三个颜色通道。
\subsection{卷积层1}

\end{document}
% 参考:
% https://blog.csdn.net/yjl9122/article/details/70198357
% https://zhuanlan.zhihu.com/p/616996325